\documentclass[a4paper, 11pt]{article}

\usepackage[a4paper, width=150mm, top=25mm, bottom=25mm, headheight=10mm]{geometry}
\usepackage{fancyhdr}
\renewcommand{\headrulewidth}{0pt}
\renewcommand{\footrulewidth}{1.2pt}
\pagestyle{fancy}
\fancyhead{}
\fancyhead[R]{\leftmark}
\fancyfoot{}
\fancyfoot[R]{page \thepage}
\fancypagestyle{sectionpagestyle}{
	\fancyhead{}
	\fancyfoot{}
	\fancyfoot[R]{page \thepage}
}

\usepackage{titlesec}
\titleformat{\section}
	{\normalfont\Large\bfseries}{\thesection}{1em}{}[{\titlerule}]

\usepackage{bookmark}
\usepackage{hyperref}
\hypersetup{
	colorlinks   = true,
	urlcolor     = black,
	linkcolor    = black,
	citecolor    = black,
	bookmarksnumbered,
	pdfencoding=auto,
	psdextra
}

\newcommand\labelthis[1]{\addtocounter{equation}{1}\phantomsection\tag{\theequation}\label{#1}}
\newcommand\mtx[1]{\begin{pmatrix} #1 \end{pmatrix}}

\usepackage{amsmath}
\usepackage{bbold}
\usepackage{tikz}
\usetikzlibrary{3d}
\usetikzlibrary{arrows.meta, arrows}

% \usepackage[active,tightpage]{preview}
% \PreviewEnvironment{tikzpicture}
% \setlength\PreviewBorder{5pt}%

\DeclareMathOperator{\colm}{colm}

\begin{document}

\begin{center}
	\Large\bfseries Linear Transformation of Cross Product
	\vspace{.15em}\hrule
\end{center}

We will prove the following identity
\begin{equation}
	M \vec a \times M \vec b = \det (M) (M^{-1})^T \vec a \times \vec b
\end{equation}
Where $M$ is a $3 \times 3$ invertible matrix. The definition of cross product leads to
\begin{equation}
	(M \vec a \times M \vec b) \cdot \vec e = \det ( \colm(M \vec a, M \vec b, \vec e ))
\end{equation}
Where $\vec e$ is a vector used for determining the components and $\colm(M \vec a, M \vec b, \hat
e)$ represents a matrix in terms of it's column vectors. Now
\begin{align*}
	\det (\colm(M \vec a, M \vec b, \vec e)) &= \det ( M \colm(\vec a, \vec b, M^{-1} \vec e ) ) \\
	&= \det (M) \det(\colm(\vec a, \vec b, M^{-1} \vec e)) \\
	&= \det (M) [ M^{-1} \vec e \cdot (\vec a \times \vec b) ] \\
	&= \det (M) (M^{-1} \vec e)^T (\vec a \times \vec b) \\
	&= \det(M) (\vec e)^T M^{-1} (\vec a \times \vec b) \\
	&= \det(M) [M^{-1} (\vec a \times \vec b)] \cdot \vec e
\end{align*}
Since this is true for every $\vec e$ we have proven the identity.

If $M$ is a rotation then it's transpose is it's inverse and has a determinant of unity. In that case
we can use the identity to show that the cross product is invariant under rotation.

\end{document}
