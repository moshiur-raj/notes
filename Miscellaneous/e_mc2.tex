%   Process with LaTeX  (Tested on LaTeX2e <1995/06/01> patch level 3)

%   This TeX encoding of the English language translation of
%   Einstein's 1905 paper:
%
%   "Ist die Tr�gheit eines K�rpers von seinem Energiegehalt abh�ngig?", Annalen der
%   Physik. 18, 639: 1905.
%
%   Was prepared by John Walker (http://www.fourmilab.ch/) from
%   the 1923 edition of "The Principle of Relativity" published
%   by Methuen and Company, Ltd.  The source document and this
%   transcription of it are in the public domain and may be used
%   in any manner without permission, restriction, attribution,
%   or compensation.

\documentclass[a4paper, 11pt]{article}
\usepackage[a4paper, width=150mm, top=25mm, bottom=25mm, headheight=10mm]{geometry}
\title{DOES THE INERTIA OF A BODY DEPEND UPON ITS ENERGY-CONTENT?}
\author{{\sc By} A. EINSTEIN}
\date{September 27, 1905}
\begin{document}
\maketitle
\renewcommand{\thefootnote}{\fnsymbol{footnote}}

The results of the previous investigation lead to a very
interesting conclusion, which is here to be deduced.

I based that investigation on the Maxwell-Hertz equations
for empty space, together with the Maxwellian expression for
the electromagnetic energy of space, and in addition the
principle that:---

{\em
The laws by which the states of physical systems alter are
independent of the alternative, to which of two systems of
coordinates, in uniform motion of parallel translation
relatively to each other, these alterations of state are
referred (principle of relativity).
}

With these principles\footnote{The principle of the constancy of the velocity of light is
of course contained in Maxwell's equations.}
as my
basis I deduced {\em inter alia} the
following result (\S~8):---

Let a system of plane waves of light, referred to the system
of co-ordinates ($x,y,z$), possess the energy $l$; let the
direction of the ray (the wave-normal) make an angle $\phi$
with the axis of $x$ of the system.  If we introduce a new system
of co-ordinates ($\xi,\eta,\zeta$) moving in uniform parallel translation
with respect to the system ($x,y,z$), and having its origin
of co-ordinates in motion along the axis of $x$ with the
velocity $v$, then this quantity of light---measured in the
system ($\xi,\eta,\zeta$)---possesses the energy

\[ l^* = l \frac{1-\frac{v}{c}cos\phi}{\sqrt{1-v^2/c^2}} \]

\noindent
where $c$ denotes the velocity of light.  We shall make use of this
result in what follows.

Let there be a stationary body in the system ($x,y,z$),
and let its energy---referred to the system ($x,y,z$) be ${\rm E}_0$.
Let the energy of the body relative to the system ($\xi,\eta,\zeta$)
moving as above with the velocity $v$, be ${\rm H}_0$.

Let this body send out, in a direction making an angle $\phi$
with the axis of $x$, plane waves of light, of energy
$\frac{1}{2}{\rm L}$ measured relatively to ($x,y,z$), and
simultaneously an equal quantity of light in the opposite
direction.  Meanwhile the body remains at rest with respect to
the system ($x,y,z$).  The principle of energy must apply to
this process, and in fact (by the principle of relativity) with
respect to both systems of co-ordinates.  If we call the energy
of the body after the emission of light ${\rm E}_1$ or
${\rm H}_1$ respectively, measured relatively to the system
($x,y,z$) or ($\xi,\eta,\zeta$) respectively, then by employing
the relation given above we obtain

\begin{eqnarray*}
    {\rm E}_0 & = & {\rm E}_1 + \frac{1}{2}{\rm L} + \frac{1}{2}{\rm L}, \\
    {\rm H}_0 & = & {\rm H}_1 + \frac{1}{2}{\rm L}
        \frac{1-\frac{v}{c}cos\phi}{\sqrt{1-v^2/c^2}}
        + \frac{1}{2}{\rm L} \frac{1+\frac{v}{c}cos\phi}{\sqrt{1-v^2/c^2}} \\
    & = & {\rm H}_1 + \frac{{\rm L}}{\sqrt{1-v^2/c^2}}.
\end{eqnarray*}

\noindent
By subtraction we obtain from these equations

\[ {\rm H}_0 - {\rm E}_0 - ({\rm H}_1 - {\rm E}_1) = {\rm L}\left\{\frac{1}{\sqrt{1-v^2/c^2}}-1\right\}. \]

\noindent
The two differences of the form ${\rm H}-{\rm E}$ occurring in
this expression have simple physical significations.  H and E
are energy values of the same body referred to two systems of
co-ordinates which are in motion relatively to each other, the
body being at rest in one of the two systems (system ($x,y,z$)).
Thus it is clear that the difference ${\rm H}-{\rm E}$ can
differ from the kinetic energy K of the body, with respect to
the other system ($\xi,\eta,\zeta$), only by an additive constant
C, which depends on the choice of the arbitrary additive
constants of the energies H and E\@. Thus we may place

\begin{eqnarray*}
    {\rm H}_0 - {\rm E}_0 & = & {\rm K}_0 + {\rm C},\\
    {\rm H}_1 - {\rm E}_1 & = & {\rm K}_1 + {\rm C},
\end{eqnarray*}

\noindent
since C does not change during the emission of light. So we have 

\[ {\rm K}_0 - {\rm K}_1 = {\rm L}\left\{ \frac{1}{\sqrt{1-v^2/c^2}}-1\right\}. \]

The kinetic energy of the body with respect to ($\xi,\eta,\zeta$)
diminishes as a result of the emission of light, and the amount
of diminution is independent of the properties of the body.
Moreover, the difference ${\rm K}_0-{\rm K}_1$, like the kinetic
energy of the electron (\S~10), depends on the velocity.

Neglecting magnitudes of fourth and higher orders we may place 

\[ {\rm K}_0 - {\rm K}_1 = \frac{1}{2}\frac{{\rm L}}{c^2}v^2. \]

\noindent
From this equation it directly follows that:---

{\em If a body gives off the energy L in the form of radiation, its mass
diminishes by ${\rm L}/c^2$.} 
The fact that the energy withdrawn from the body
becomes energy of radiation evidently makes no difference, so that we
are led to the more general conclusion that

The mass of a body is a measure of its energy-content; if the energy
changes by L, the mass changes in the same sense by
${\rm L}/9\times 10^{20}$, the energy being measured
in ergs, and the mass in grammes.

It is not impossible that with bodies whose energy-content is variable
to a high degree (e.g.\ with radium salts) the theory may be
successfully put to the test.

If the theory corresponds to the facts, radiation conveys 
inertia between the emitting and absorbing bodies.

\vfill

\begin{quotation}
{\small
\begin{center}
{\sc About this Document}
\end{center}
\vspace{2ex}
{\sf
This edition of Einstein's {\em Does the Inertia of a Body
Depend upon its Energy-Content} is based on the English translation of his original
1905 German-language paper (published as
{\em Ist die Tr\"agheit eines K\"orpers von seinem Energiegehalt abh\"angig?},
in {\em Annalen der Physik}. {\bf 18}:639, 1905) which appeared in
the book {\em The Principle of Relativity}, published in 1923 by
Methuen and Company, Ltd.\ of London.  Most of the papers in
that collection are English translations by W. Perrett and G.B.
Jeffery from the German {\em Das Relativatsprinzip}, 4th ed.,
published by in 1922 by Tuebner.  All of these sources are now
in the public domain; this document, derived from them, remains
in the public domain and may be reproduced in any manner or
medium without permission, restriction, attribution, or
compensation.

The footnote is as it appeared in the 1923 edition.  The 1923 English
translation modified the notation used in Einstein's 1905 paper to
conform to that in use by the 1920's; for example, $c$ denotes
the speed of light, as opposed the V used by Einstein in 1905.
In this paper Einstein uses L to denote energy; the
italicised sentence in the
conclusion may be written as the equation ``$m={\rm L}/c^2$'' which,
using the more modern E instead of L to denote energy, may be
trivially rewritten as ``${\rm E}=mc^2$''.

This edition was prepared by John Walker.  The current
version of this document is available in a variety of
formats from the editor's Web site:
\begin{center}
{\tt http://www.fourmilab.ch/}
\end{center}
}
}
\end{quotation}

\end{document}
