\documentclass[titlepage, a4paper, 11pt]{article}

\usepackage{amsmath}

\usepackage[a4paper, width=150mm, top=25mm, bottom=25mm, headheight=10mm]{geometry}

\begin{document}

\title{Conservation of Linear and Angular Momentum in Magnetostatics}
\author{Moshiur Rahman}
\date{\today}

\begin{abstract}
	Momentum conservation for an isolated system is a fundamental law of physics. As such it applies
	to both electrostatics and magnetostatics. Conservation principles for electrostatics is
	often discussed in classes. However it is almost always avoided for magnetostatics. But the case
	for magnetostatics is just as important for understanding the physics behind the technologies
	(e.g. electric motors) we rely on everyday. In electrostatics conservation of linear and angular
	momentum follows quite easily from the fact that Coulomb's law obeys Newton's third law and acts
	radially. It can also be shown using Lagrangian mechanics. However for magnetostatics these
	methods do not work as magnetic force does not act radially and constructing the Lagrangian for
	a system of magnetically interacting particles is not as simple. Unlike electrostatics we cannot
	reduce the problem to the interaction between two charged particles, we have to consider the
	interactions in the entire system. Which makes this an interesting problem to solve. An approach
	is shown here using the Biot-Savant law.
\end{abstract}

\maketitle

\section{Introduction}
Magnetostatics refers to the situations where the electric charge density and the electric current
density at every point in space does not change with time. In such cases the magnetic field at a
point $\mathbf{r}$ is given by the Biot-Savant law \cite{Griffiths} in terms of the electric current
density $\mathbf{J}(\mathbf{r})$.
\begin{equation}
	\mathbf{B}(\mathbf{r}) = \frac{\mu_0}{4\pi} \int_V \frac{\mathbf{J}(\mathbf{r}') \times
	(\mathbf{r} - \mathbf{r}')}{|\mathbf{r} - \mathbf{r}'|^3} d^3\mathbf{\mathbf{r}}'
	\label{Biot-Savant law}
\end{equation}
Where the integration region, $V$, contains all the charges in the system and these charges
experience the Lorentz force given by
\begin{equation}
	\mathbf{F} = q (\mathbf{E} + \mathbf{v} \times \mathbf{B})
\end{equation}
We can deal with the electric and magnetic force separately. The conservation law for the electric
force is easy to demonstrate. For most cases we deal with electrically neutral systems anyway. In
this article we will only do calculations for the magnetic field.

To demonstrate the conservation of linear and angular momentum it is enough to prove that the net
force and the net torque is zero. Assuming a continuous charge distribution ($\rho$) and using the
Lorentz force law along with the identity $\mathbf{J} = \rho \mathbf{v}$, we can derive the
expression for the net force\cite{Griffiths} and the net torque.
\begin{align}
	\mathbf{F}_\text{net} &= \int_V \mathbf{J}(\mathbf{r}) \times \mathbf{B}(\mathbf{r}) d^3\mathbf{r}
	\label{F_net} \\
	\mathbf{T}_\text{net} &= \int_V \mathbf{r} \times \left[ \mathbf{J}(\mathbf{r}) \times
	\mathbf{B}(\mathbf{r}) \right] d^3 \mathbf{r}
	\label{T_net}
\end{align}

\section{Conservation of Linear Momentum}
Using the Biot-Savant law the expression for the net force Eq. \eqref{F_net}
becomes
\begin{equation}
	\mathbf{F}_\text{net} = \frac{\mu_0}{4\pi} \int \int \mathbf{J}(\mathbf{r}) \times \left[
	\mathbf{J}(\mathbf{r}') \times \frac{\mathbf{r} - \mathbf{r}'}{| \mathbf{r} - \mathbf{r}'
	|^3} \right] d^3\mathbf{r}' d^3\mathbf{r}
\end{equation}
We can use the vector triple product formula here. $\mathbf{A} \times (\mathbf{B} \times \mathbf{C})
= (\mathbf{A} \cdot \mathbf{C}) \mathbf{B} - (\mathbf{A} \cdot \mathbf{B}) \mathbf{C}$.
\begin{equation}
	\mathbf{F}_\text{net} = \frac{\mu_0}{4\pi} \int \int \left[ \left\{ \mathbf{J}(\mathbf{r}) \cdot
	\frac{\mathbf{r} - \mathbf{r}'}{| \mathbf{r} - \mathbf{r}' |^3} \right\} \mathbf{J}(\mathbf{r}')
	+ \left\{ \mathbf{J}(\mathbf{r}) \cdot \mathbf{J}(\mathbf{r}') \right\} \frac{\mathbf{r} -
	\mathbf{r}'}{| \mathbf{r} - \mathbf{r}' |^3} \right] d^3\mathbf{r}' d^3\mathbf{r}
\end{equation}
The  second term inside the integral changes sign when $\mathbf{r}$ and $\mathbf{r}'$ are
interchanged. This implies that the contribution of the term evaluated at $(\mathbf{r},
\mathbf{r}')$ is exactly canceled out by the contribution from $(\mathbf{r}', \mathbf{r})$. So the
integral of that term is zero. The first term can be rewritten using the following relation.
\begin{equation}
	- \nabla \frac{1}{|\mathbf{r} - \mathbf{r}'|} = \frac{\mathbf{r} - \mathbf{r}'}{| \mathbf{r} -
	\mathbf{r}' |^3}
	\label{grad 1/r}
\end{equation}
Then the net force becomes
\begin{equation}
	\mathbf{F}_\text{net} = - \frac{\mu_0}{4\pi} \int \left[ \int \mathbf{J}(\mathbf{r}) \cdot
	\nabla \frac{1}{|\mathbf{r} - \mathbf{r}'|} d^3\mathbf{r} \right] \mathbf{J}(\mathbf{r}') \,
	d^3\mathbf{r}'
\end{equation}
For a scalar field $f$ and a divergenceless vector field $\mathbf{A}$, $\nabla \cdot (f \mathbf{A})
= \nabla f \cdot \mathbf{A}$. Since the charge density does not change with time, the continuity
equation\cite{Griffiths} implies that $\mathbf{J}$ is divergenceless. Hence
\begin{equation}
	\nabla \cdot \left[ \frac{1}{|\mathbf{r} - \mathbf{r}'|} \mathbf{J}(\mathbf{r}) \right] = \nabla
	\frac{1}{|\mathbf{r} - \mathbf{r}'|} \cdot \mathbf{J}(\mathbf{r})
	\label{div 1/|r-r'| J}
\end{equation}
And
\begin{equation}
	\mathbf{F}_\text{net} = - \frac{\mu_0}{4\pi} \int \left[ \int \nabla \cdot \left\{
	\frac{1}{|\mathbf{r} - \mathbf{r}'|} \mathbf{J}(\mathbf{r}) \right\} d^3\mathbf{r} \right]
	\mathbf{J}(\mathbf{r}') \, d^3\mathbf{r}'
\end{equation}
We can convert this integral inside the third brackets to a surface integral using the divergence
theorem. But we can make the volume arbitrarily large and for a sufficiently large volume
$\mathbf{J}(\mathbf{r}) = \mathbf{0}$ at the surface. So, the surface integral ultimately becomes
zero. Implying
\begin{equation}
	\mathbf{F}_\text{net} = \mathbf{0}
\end{equation}

\section{Conservation of Angular Momentum}
Again using the Biot-Savant law the expression for the net torque Eq. \eqref{T_net} becomes
\begin{equation}
	\mathbf{T}_\text{net} = \int \int \mathbf{r} \times \left[ \mathbf{J}(\mathbf{r}) \times
	\left\{ \mathbf{J}(\mathbf{r}') \times \frac{\mathbf{r} - \mathbf{r'}}{|\mathbf{r} -
	\mathbf{r}'|^3} \right\} \right] d^3 \mathbf{r}' d^3 \mathbf{r}
\end{equation}
Using the vector triple product formula and expanding the terms we arrive at
\begin{equation}
	\mathbf{T}_\text{net} = \int \int \left[ \left\{ \mathbf{J}(\mathbf{r}) \cdot \frac{\mathbf{r} -
	\mathbf{r'}}{|\mathbf{r} - \mathbf{r}'|^3} \right\} \mathbf{r} \times \mathbf{J}(\mathbf{r}') +
	\left\{ \mathbf{J}(\mathbf{r}) \cdot \mathbf{J}(\mathbf{r}') \right\} \frac{\mathbf{r} \times
	\mathbf{r'}}{|\mathbf{r} - \mathbf{r}'|^3} \right] d^3 \mathbf{r}' d^3 \mathbf{r}
\end{equation}
The integral of the second term is zero as interchanging $\mathbf{r}$ and $\mathbf{r}'$ changes the
sign. And using Eq. \eqref{grad 1/r}, Eq. \eqref{div 1/|r-r'| J} for the first term the net torque becomes
\begin{equation}
	\mathbf{T}_\text{net} = - \int \left[ \int \left\{ \nabla \cdot
	\frac{\mathbf{J}(\mathbf{r})}{|\mathbf{r} - \mathbf{r}'|} \right\} \mathbf{r} d^3 \mathbf{r}
	\right] \times \mathbf{J}(\mathbf{r}') d^3 \mathbf{r}'
	\label{T_net somewhat simplified}
\end{equation}
The integral inside the third brackets is a vector. We will evaluate its components separately.
\begin{equation}
	\begin{aligned}
		\int \left\{ \nabla \cdot \frac{\mathbf{J}(\mathbf{r})}{|\mathbf{r} - \mathbf{r}'|} \right\}
		x d^3\mathbf{r} &= \int \left\{ \frac{\partial}{\partial x}
		\frac{J_x(\mathbf{r})}{|\mathbf{r} - \mathbf{r}'|} \right\} x d^3\mathbf{r} + \int \left\{
		\frac{\partial}{\partial y} \frac{J_y(\mathbf{r})}{|\mathbf{r} - \mathbf{r}'|} \right\}
		x d^3\mathbf{r} \\
		&+ \int \left\{ \frac{\partial}{\partial z} \frac{J_z(\mathbf{r})}{|\mathbf{r} -
		\mathbf{r}'|} \right\} x d^3\mathbf{r}
	\end{aligned}
	\label{int div J/|r-r'| x d^3r}
\end{equation}
The first term can be rewritten using integration by parts.
\begin{equation}
	\int \left\{ \frac{\partial}{\partial x} \frac{J_x(\mathbf{r})}{|\mathbf{r} - \mathbf{r}'|}
	\right\} x dx = \int \left[ \frac{\partial}{\partial x} \left\{ \frac{xJ_x(\mathbf{r})}{
	|\mathbf{r} - \mathbf{r}'|} \right\} - \frac{J_x(\mathbf{r})}{|\mathbf{r} - \mathbf{r}'|}
	\right] dx
\end{equation}
But, $\mathbf{J}(\mathbf{r})$ vanishes at the boundary for sufficiently large volume. Meaning the
boundary term vanishes.
\begin{equation}
	\int \left\{ \frac{\partial}{\partial x} \frac{J_x(\mathbf{r})}{|\mathbf{r} - \mathbf{r}'|}
	\right\} x d^3\mathbf{r} = - \int \frac{J_x(\mathbf{r})}{|\mathbf{r} - \mathbf{r}'|} d^3
	\mathbf{r}
\end{equation}
The second term can be evaluated in a similar way.
\begin{equation}
	\int \left\{ \frac{\partial}{\partial y} \frac{J_y(\mathbf{r})}{|\mathbf{r} - \mathbf{r}'|}
	\right\} x dy = \int \frac{\partial}{\partial y} \left\{ x \frac{J_y(\mathbf{r})}{|\mathbf{r} -
	\mathbf{r}'|} \right\} dy = 0
\end{equation}
Implying
\begin{equation}
	\int \left\{ \frac{\partial}{\partial y} \frac{J_y(\mathbf{r})}{|\mathbf{r} - \mathbf{r}'|}
	\right\} x d^3 \mathbf{r} = 0
\end{equation}
Likewise the third term is also zero.
\begin{equation}
	\int \left\{ \frac{\partial}{\partial z} \frac{J_z(\mathbf{r})}{|\mathbf{r} - \mathbf{r}'|}
	\right\} x d^3 \mathbf{r} = 0
\end{equation}
Then Eq. \eqref{int div J/|r-r'| x d^3r} reduces to
\begin{equation}
	\int \left\{ \nabla \cdot \frac{\mathbf{J}(\mathbf{r})}{|\mathbf{r} - \mathbf{r}'|} \right\} x
	d^3\mathbf{r} = - \int \frac{J_x(\mathbf{r})}{|\mathbf{r} - \mathbf{r}'|} d^3 \mathbf{r}
\end{equation}
The $y$ and $z$ components can be evaluated correspondingly.
\begin{align}
	\int \left\{ \nabla \cdot \frac{\mathbf{J}(\mathbf{r})}{|\mathbf{r} - \mathbf{r}'|} \right\} y
	d^3\mathbf{r} = - \int \frac{J_y(\mathbf{r})}{|\mathbf{r} - \mathbf{r}'|} d^3 \mathbf{r} \\
	\int \left\{ \nabla \cdot \frac{\mathbf{J}(\mathbf{r})}{|\mathbf{r} - \mathbf{r}'|} \right\} z
	d^3\mathbf{r} = - \int \frac{J_z(\mathbf{r})}{|\mathbf{r} - \mathbf{r}'|} d^3 \mathbf{r}
\end{align}
Combining the three equations above, Eq. \eqref{T_net somewhat simplified} simplifies to
\begin{equation}
	\mathbf{T}_\text{net} = \int \int \frac{\mathbf{J}(\mathbf{r})}{|\mathbf{r} - \mathbf{r}'|} \times
	\mathbf{J}(\mathbf{r}') d^3\mathbf{r} d^3\mathbf{r}'
\end{equation}
This integral is zero as the function inside the integral changes sign when $\mathbf{r}$ and
$\mathbf{r}'$ are interchanged.
\begin{equation}
	\mathbf{T}_\text{net} = \mathbf{0}
\end{equation}

\section{Remarks}
We have successfully proven that conservation of linear and angular momentum holds true for
magnetostatics. But how does it help us understand the technologies we use everyday? Well, think of
an electric motor. When the motor generates torque, is there an equal and opposite reaction
involved? The motor is not exactly a magnetostatic system as the loop of wires inside the motor
rotates, creating a time-varying current density. But we can approximate it as a system where
Biot-Savant law is applicable. This is because we can ignore the relativistic effects as the time it
takes for light to traverse the body of the motor is negligibly small compared to the time we take
to observe the system. To be specific we can approximate the retarded time as the real time
\cite{Griffiths}. The continuity equation also holds as the wire is an electrically neutral system.
And the magnetic field of the magnet can be approximated as originating from microscopic current
loops. Thus we can say that there is indeed an equal and opposite torque applied to the support that
holds the motor. Our proof also demonstrates that when a magnet is repelled or pulled by another
magnet linear and angular momentum of the entire system is conserved.

\section{Author Declaration Section}
The authors have no conflicts to disclose.

\begin{thebibliography}{1}
	\bibitem{Griffiths} David J. Griffiths, \textit{Introduction to Electrodynamics}, 4th Ed. (Cambridge
		University Press, 2017), p.~231, 221, 222, 445.
\end{thebibliography}

\end{document}
