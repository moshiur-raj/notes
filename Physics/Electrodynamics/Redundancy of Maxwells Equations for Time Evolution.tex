\documentclass[a4paper, 11pt]{article}

\usepackage[a4paper, width=150mm, top=25mm, bottom=25mm, headheight=10mm]{geometry}
\usepackage{fancyhdr}
\renewcommand{\headrulewidth}{0pt}
\renewcommand{\footrulewidth}{1.2pt}
\pagestyle{fancy}
\fancyhead{}
\fancyhead[R]{\leftmark}
\fancyfoot{}
\fancyfoot[R]{page \thepage}
\fancypagestyle{sectionpagestyle}{
	\fancyhead{}
	\fancyfoot{}
	\fancyfoot[R]{page \thepage}
}

\usepackage{titlesec}
\titleformat{\section}
	{\normalfont\Large\bfseries}{\thesection}{1em}{}[{\titlerule}]

\usepackage{bookmark}
\usepackage{hyperref}
\hypersetup{
	colorlinks   = true,
	urlcolor     = black,
	linkcolor    = black,
	citecolor    = black,
	bookmarksnumbered,
	pdfencoding=auto,
	psdextra
}

\usepackage{amsmath}

\begin{document}

\begin{center}
	\Large\bfseries Redundancy of Maxwell's Equations For Time Evolution
	\vspace{.15em}\hrule
\end{center}

The four Maxwell's equations are,
\begin{alignat}{2}
	&\nabla \cdot E &&= \rho / \epsilon_0 \label{div E} \\
	&\nabla \cdot B &&= 0 \label{div B} \\
	&\nabla \times E &&= - \frac{\partial B}{\partial t} \label{curl E} \\
	&\nabla \times B &&= \mu_0 J + \mu_0 \epsilon_0 \frac{\partial E}{\partial t} \label{curl B}
\end{alignat}
Together they constitute eight differential equations. However two of them are only needed for
setting the initial condition and they are not necessary for time evolution even without using
potentials to describe the field. \medskip

If we know the value of the electromagnetic field at a particular time then we can evolve the field
in time using \eqref{curl E} and \eqref{curl B}. Let, $E = E_0$, $B = B_0$, $\rho = \rho_0$ and $J =
J_0$ at time $t = t_0$.
Then,
\begin{align}
	E(\vec{r}, t_0 + dt) &= E_0 + \frac{\partial E}{\partial t} = E_0 + \left( \frac{1}{\mu_0
	\epsilon_0} \nabla \times B - \frac{1}{\epsilon_0} J \right) dt \\
	B(\vec{r}, t_0 + dt) &= B_0 + \frac{\partial B}{\partial t} dt = B_0 - \nabla \times E \, dt
\end{align}
Now we take the divergence of the electric and magnetic field at $t = t_0 + dt$.
\begin{align}
	\nabla \cdot E(\vec{r}, t_0 + dt) &= \nabla \cdot E_0 - \frac{1}{\epsilon_0} \nabla \cdot J \, dt
	\\
	\nabla \cdot B(\vec{r}, t_0 + dt) &= \nabla \cdot B_0
\end{align}
We have used the fact that divergence of curl is zero. Maxwell's fourth equation \eqref{curl B}
implies the continuity equation for charge flow e.g. $\nabla \cdot J = - \partial \rho / \partial t$
. The electromagnetic field at time $t = t_0$ satisfies the Maxwell's equations. Hence the two
equations above reduces to,
\begin{align}
	\nabla \cdot E(\vec{r}, t_0 + dt) &= \frac{1}{\epsilon_0} \rho_0 + \frac{1}{\epsilon_0} \frac{
	\partial \rho}{\partial t} \, dt = \frac{1}{\epsilon_0} \rho(\vec{r}, t_0 + dt) \\
\nabla \cdot B(\vec{r}, t_0 + dt) &= 0
\end{align}

This means that if \eqref{div E} and \eqref{div B} are satisfied at $t = t_0$ then infinitesimal
evolution of the field using \eqref{curl E} and \eqref{curl B} will also satisfy \eqref{div E} and
\eqref{div B}. We can construct the evolution of the field for finite time difference by integrating
the infinitesimal changes, hence the divergence of the field always follows Maxwell's equations if
the initial conditions do the same. \medskip

From this discussion we can conclude that \eqref{div E} and \eqref{div B} are only needed for the
initial condition. Once the initial value of the field is fixed we only need two of the Maxwell's
equations to evolve the field, hence the redundancy. \bigskip

We can approach the problem in a much more rigorous and compact way given below.
\begin{alignat}{2}
	\frac{\partial}{\partial t} \nabla \cdot E &= \nabla \cdot \frac{\partial}{\partial t} E &&=
	\nabla \cdot \left( \frac{1}{\mu_0 \epsilon_0} \nabla \times B - \frac{1}{\epsilon_0} J \right)
	= \frac{1}{\epsilon_0} \frac{\partial \rho}{\partial t} \\
	\frac{\partial}{\partial t} \nabla \cdot B &= \nabla \cdot \frac{\partial}{\partial t} B &&=
	\nabla \cdot \left( - \nabla \times E \right) = 0
\end{alignat}
These equations imply that $\nabla \cdot E - \rho/\epsilon_0$ and $\nabla \cdot B$ are constants.
These constants are determined by the initial condition. However this approach is less intuitive.

\end{document}
