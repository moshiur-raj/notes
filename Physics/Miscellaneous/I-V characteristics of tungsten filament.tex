\documentclass[titlepage, a4paper, 11pt]{article}

\usepackage[a4paper, width=150mm, top=25mm, bottom=25mm, headheight=10mm]{geometry}
\usepackage{fancyhdr}
\renewcommand{\headrulewidth}{0pt}
\renewcommand{\footrulewidth}{1.2pt}
\pagestyle{fancy}
\fancyhead{}
\fancyhead[R]{\leftmark}
\fancyfoot{}
% \fancyfoot[R]{page \thepage}
\fancypagestyle{sectionpagestyle}{
	\fancyhead{}
	\fancyfoot{}
	% \fancyfoot[R]{page \thepage}
}

\usepackage{amsmath}

\begin{document}

Let us assume a tungsten filament has uniform cross sectional area $A$, length $L$ and total surface
area $S = 2A + A'$ ($A'$ is the remaining surface area after excluding the two ends). The resistance
of tungsten varies linearly with temperature. This is valid for very high temperatures (in the order
of $10^3K$).
\begin{equation}
	R = R_0 [1 + \alpha (T - T_0)]
	\label{r in terms of t}
\end{equation}
Here, $T_0$ is the room temperature and $T$ is the temperature of the filament and $R_0$ can be
rewritten in terms of resistivity $\rho$, $A$ and $L$ as,
\begin{equation}
	R_0 = \rho \, \frac{L}{A}
	\label{r0 in terms of rho}
\end{equation}

When electricity flows through the filament it heats up. According to the Stefan – Boltzmann law the
energy radiated will be,
\begin{equation}
	P_\text{ heat} = \sigma S (T^4 - T_0^4)
	\label{p heat}
\end{equation}
The energy supplied to the filament through electricity is,
\begin{equation}
	P_\text{ electricity} = I V
	\label{p electricity}
\end{equation}

At equilibrium the energy supplied by electricity is spent in radiated heat. Hence,
\begin{equation}
	I V = \sigma S (T^4 - T_0^4)
	\label{equilibrium}
\end{equation}
Now, we only have to express the temperature as function of current and volatage in the above
equation.

Combining \eqref{r in terms of t}, \eqref{r0 in terms of rho} and using Ohm's law,
\begin{equation}
	R = \frac{V}{I} = \rho \, \frac{L}{A} \, [1 + \alpha (T - T_0)]
	\label{r in terms of v i and rho}
\end{equation}

The tungsten filament gets very hot (order of $10^3K$). So we can ignore $T_0$ and at this
temperature range we can assume $\alpha T \gg 1$ (for tungsten). Then
\eqref{r in terms of v i and rho} simplifies to,
\begin{equation}
	\frac{V}{I} = \rho \, \frac{L}{A} \, \alpha T \implies T = \frac{1}{\rho \alpha} \, \frac{A}{L}
	\, \frac{V}{I}
\end{equation}
Putting the value of $T$ in \eqref{equilibrium} and ignoring $T_0$ we arrive at,
\begin{equation}
	I V = \sigma \, (2A + A') \left( \frac{1}{\rho \alpha} \, \frac{A}{L} \, \frac{V}{I} \right)^4
\end{equation}
Rearranging the terms,
\begin{equation}
	I = \left[ \frac{\sigma}{(\rho \alpha)^4} \right]^{1/5} \left[ (2A + A') A^4 \right]^{1/5}
	\frac{1}{L^{4/5}} \, V^{3/5} = k V^n
	\label{i = kv^n}
\end{equation}

Hence, $I$ should vary linearly with $V^{0.6}$. Experimentally I found that the value of $n$ was
$0.5$.

The above equation also demonstrates the dependence of $k$ on the cross-section and length of the
conductor. This behavior is different from that of conductance. Conductance varies linearly with
$A/L$. It is due to the fact that if we change the area or length we cannot assume that
the configuration is similar to filaments in parallel or series respectively, as we need to
consider the outer surface area for Stefan - Boltzmann law. If the length of the filament is
small compared to it's thickness then we can ignore the effects of surface area (ignoring $A'$ in
\eqref{i = kv^n}) when changing the cross section and we recover the linear dependence on $A$.
However, the linear dependence on $1/L$ is not recovered as we cannot ignore the effect of surface
area when changing $L$.

\end{document}
