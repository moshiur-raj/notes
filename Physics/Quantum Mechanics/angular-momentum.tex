\documentclass[a4paper, 12pt]{article}

\setlength{\parindent}{0pt}
\renewcommand{\indent}{\hspace{3ex}}

\usepackage[top=1in,bottom=1in,left=1.25in,right=1.25in]{geometry}
% \usepackage[document]{ragged2e}
\usepackage{fancyhdr}
\renewcommand{\headrulewidth}{0pt}
\renewcommand{\footrulewidth}{1.2pt}
\pagestyle{fancy}
\fancyhead{}
\fancyfoot{}
\fancyfoot[R]{page \thepage}

\renewcommand{\maketitle}{
	\begin{center}
		\Huge\bfseries{Intuition Behind Angular Momentum}
	\end{center}
	\bigskip
}

\usepackage{titlesec}
\titleformat{\section}
	{\normalfont\Large\bfseries}{\thesection}{1em}{}[{\titlerule}]
\titleformat{\subsection}
	{\normalfont\large\bfseries}{$\bullet$}{1ex}{}

\usepackage{hyperref}
\hypersetup{
  colorlinks   = true,
  urlcolor     = blue,
  linkcolor    = blue,
  citecolor    = red
}

\usepackage{amsmath}

\begin{document}
\maketitle

\section{The Problem}
\indent Spherical Harmonics are the eigenfunctions of the square of the angular momentum operator defined as $ \hat L = \vec r \times \hat p$. For a spherically symmetric potential we can express the wavefunction as a product of the spherical harmonics and the radial wave equation. However, the eigenfunctions of the hamiltonian will have stationary probability density. This is because the solutions are of the form $\Psi = R_n^l (r) \, Y_l^m (\theta, \phi) \exp(-i \omega t)$. If we take $|\Psi|^2$ the time dependent parts cancel out. But the eigenstate will still have angular momentum with $\langle {\hat L}^2 \rangle = l (l+1) \hbar^2$ and $\langle \hat L_z \rangle = m \hbar$. How can there be angular momentum if the probability density is constant in time at every point in space? Does it not seem like nothing is moving/changing?

\section{The Solution: Probability Current}
\indent Constant probability density can still mean that there's a probability current present if the probability coming inside a volume is equal to the probability going out. Let us define the probability current the following way.
\begin{equation}
	\oint_s \vec J \cdot d\vec S = - \frac{d}{dt} \int_\tau |\Psi|^2 d\tau \label{definition of J}
\end{equation}

This simply means that the change in probability in a closed volume comes from the flow of probability current. Using the divergence theorem we can turn the surface integral into a volume integral. And using Schrodinger's equation for a single particle we can turn the time derivative of $\Psi$ and $\Psi^*$ to spacial derivatives. Doing these and comparing the terms inside the volume integral we get,
\begin{equation}
	\nabla \cdot \vec J = \frac{i \hbar}{2M} \left( \Psi \nabla^2 \Psi^* - \Psi^* \nabla^2 \Psi \right) = \nabla \cdot \frac{i \hbar}{2M} \left( \Psi \nabla\Psi^* - \Psi^* \nabla\Psi \right) \label{divergence of J}
\end{equation}

From \eqref{divergence of J} it makes sense to assume that,
\begin{equation}
	\vec J = \frac{i \hbar}{2M} \left( \Psi \nabla\Psi^* - \Psi^* \nabla\Psi \right) \label{expression for J}
\end{equation}

However, we can add any divergenceless field (say $\vec g$ where $\nabla \cdot \vec g = 0$) to this $\vec J$ and it would still satisfy \eqref{divergence of J}. Then why should we think \eqref{expression for J} is a good expression for evaluating $\vec J$ ?

\subsection{Analogy From Fluid Dynamics}
\indent In fluid Dynamics the current density the magnitude of $\vec J$ represents the mass of particles passing through an unit area of cross section per unit time. In particular, $\vec J = \rho \vec v$. Where $\vec v$ is the direction of fluid flow at a point and $\rho$ is the density of the fluid. Now, this J also satisfies a continuity equation like \eqref{definition of J}. So we can compare this current density with the probability current. We simply have to replace $\rho$ with $ M \, \Psi^* \Psi$.

\indent If we want to know the average momentum of a given volume of fluid then we have to take the volume integral of $\vec J$. As,
\begin{equation}
	\langle \vec p \rangle = \int_\tau \vec v \, dm = \int_\tau \vec v \rho \, d\tau = \int_\tau \vec J \, d\tau \label{expression for average momentum}
\end{equation}

In the quantum case, let us replace the $\nabla$ operator in \eqref{expression for J} with the linear momentum operator $\hat p$.
\begin{equation}
	\vec J = \frac{1}{2M} \left( \Psi \, \hat p \Psi^* + \Psi^* \, \hat p \Psi \right) \label{J in terms of p}
\end{equation}
This looks very similar to the expression for determining the average value of $\hat p$. We are only missing a volume integral. Hence,
\begin{equation}
	\int_\tau \vec J \, d\tau = \frac{1}{2M} 2 \langle \hat p \rangle =  \frac{\langle \hat p \rangle} {M}
\end{equation}
This is consistent with the intuition from fluid dynamics. If we had added a divergenceless field to $\vec J$ then this result would not be true. Thus \eqref{expression for J} makes sense.

\section{Mathematical Proof}
\indent Let us now interpret the probability current as a mass flow. Then the angular momentum is simply given by this volume integral,
\begin{equation}
	\langle \hat L \rangle = \int_\tau \vec r \times M \vec J \, d\tau
\end{equation}

Using \eqref{J in terms of p} we get,
\begin{align*}
	\langle \hat L \rangle &= \frac{1}{2} \int_\tau \vec r \times \left( \Psi \, \hat p \Psi^* + \Psi^* \, \hat p \Psi \right) \, d\tau \\
						   &= \frac{1}{2} \int_\tau \left( \Psi \vec r \times \hat p \Psi^* + \Psi^* \vec r \times \hat p \Psi \right) \, d\tau \\
\end{align*}
Now using the fact that $\vec r \times \hat p$ is a hermitian operator we finally get,
\[ \langle \hat L \rangle = \langle \vec r \times \hat p \rangle \]
This means that the definition of $\hat L = \vec r \times \hat p$ aligns perfectly with our intuition from fuild dynamics. We can interpret the probability current as a mass flow that gives rise to angular momentum even in stationary states (eigenstates).

\indent Using this method we can also determine the dipole magnetic moment. We simply have to get the electrical current density using $-e \vec J$. Then using the classical definition of dipole moment for a current density,
\begin{align*}
	\mu &= \frac{1} {2} \int_\tau \vec r \times -e \vec J \, d\tau \\
		&= - \frac{e}{2M} \int_\tau \vec r \times M \vec J , d\tau \\
		&= - \frac{e}{2M} \langle \hat L \rangle \\
\end{align*}

\section{Conclusion}
\indent Using the expression for probability current we can get classical intuitions for angular momentum and the dipole moment for eigenstates. But what does $\vec J$ mean in terms of Quantum Mechanics? Is anything actually flowing? Electron is not a mass/charge distribution but a point particle so our analogy does not go very far. How do we reconcile our calculations with the born interpretation?

\end{document}
