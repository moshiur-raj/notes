\documentclass[a4paper, 11pt]{article}

\usepackage[a4paper, width=150mm, top=25mm, bottom=25mm, headheight=10mm]{geometry}
\usepackage{fancyhdr}
\renewcommand{\headrulewidth}{0pt}
\renewcommand{\footrulewidth}{1.2pt}
\pagestyle{fancy}
\fancyhead{}
\fancyhead[R]{\leftmark}
\fancyfoot{}
\fancyfoot[R]{page \thepage}
\fancypagestyle{sectionpagestyle}{
	\fancyhead{}
	\fancyfoot{}
	\fancyfoot[R]{page \thepage}
}

\usepackage{titlesec}
\titleformat{\section}
	{\normalfont\Large\bfseries}{\thesection}{1em}{}[{\titlerule}]

\usepackage{bookmark}
\usepackage{hyperref}
\hypersetup{
	colorlinks   = true,
	urlcolor     = black,
	linkcolor    = black,
	citecolor    = black,
	bookmarksnumbered,
	pdfencoding=auto,
	psdextra
}

\usepackage{amsmath}

\title{Momentum Space Wavefunction}
\author{}
\date{}

\begin{document}

\maketitle

A n-body wavefunction is represented by the function,
\begin{equation}
	\Psi(\vec{r}_1, \vec{r}_2, \ldots, \vec{r}_N, t)
\end{equation}

We will now impose the boundary condition that the particles are all in a box and then we will take
the limit as the size of the box goes to infinity. Let the box, $\mathcal{B}$, be defined by the conditions,
\begin{equation}
\begin{aligned}
	0 < x - x_0 < L_x \\
	0 < y - y_0 < L_y \\
	0 < z - z_0 < L_z \\
\end{aligned}
\end{equation}
Here $x_0, y_0, z_0$ and $L_1, L_2, L_3$ are constants. The volume of the box is, $V = L_x L_y L_z$.

Now the fourier series of the wavefunction is given by,
\begin{align}
	\Psi &= \sum_{n_1, n_2, \ldots, n_N = -\infty}^{+\infty} C_{n_1, n_2, \ldots, n_N} \exp \left[ i \sum_{j = 1}^N 2\pi n_j \left( \frac{x_j}{L_x} + \frac{y_j}{L_y} + \frac{z_j}{L_z}\right) \right] \\
	&= \sum_{n_1 , n_2, \ldots, n_N = -\infty}^{+\infty} C_{n_1, n_2, \ldots, n_N} \exp \left[ i \sum_{j = 1}^N \vec{k}_{n_j} \cdot \vec{r_j} \right]
\end{align}
Where,
\begin{equation}
	C_{n_1, n_2, \ldots, n_N} = \frac{1}{V^N} \int_{\mathcal{B}^N} \Psi \exp \left[ -i \sum_{j =
	1}^N \vec{k}_{n_j} \cdot \vec{r_j} \right] d\tau_r
\end{equation}

The momentum operator for the $j$'th particle is defined as,
\begin{equation}
	\hat{p}_j = -i \hbar \nabla_j
\end{equation}
This operator has eigenfunctions of the form $\exp( \frac{i}{\hbar} \vec{p}_{n_j} \cdot \vec{r} )$
for an eigenvalue $\vec{p}_{n_j}$. The total wavefunction is then,
\begin{equation}
	f_{n_1, n_2, \ldots, n_N} = A \prod_{j = 1}^N \exp( \frac{i}{\hbar} \vec{p}_{n_j} \cdot
	\vec{r}_j )
\end{equation}
Normalizing the wavefunction in the region $\mathcal{B}^N$ we get, $A = V^{-N/2}$.

We can now express $\Phi$ as a superposition of normalized momentum states.
\begin{equation}
	\Psi = \sum C_{n_1, n_2, \ldots, n_N} V^{N/2} f_{n_1, n_2, \ldots, n_N}
\end{equation}
Where,
\begin{equation}
	\vec{p}_{n_j} = \hbar \vec{k}_{n_j}
\end{equation}

Now,
\begin{equation}
	d \vec{p}_{n_j} = \vec{p}_{n_j+1} - \vec{p}_{n_j} = 2\pi \left( 1/L_x, 1/L_y, 1/L_z \right)
\end{equation}
Hence,
\begin{equation}
	d^3 \vec{p}_{n_j} = (2\pi)^3 / V
\end{equation}
Thus,
\begin{equation}
	d\tau_p = \prod d^3 \vec{p}_{n_j} = (2\pi)^{3N} / V^N
\end{equation}

The probability of measuring the momentum of the particles as $\vec{p}_{n_1}, \vec{p}_{n_2}, \ldots,
 \vec{p}_{n_j}$ respectively is then,
 \begin{align*}
	 \left| \sum C_{n_1, n_2, \ldots, n_N} V^{N/2} \right|^2 &= \frac{1}{V^N} \left|
	 \int_{\mathcal{B}^N} \Psi \exp \left[ -i \sum_{j = 1}^N \vec{k}_{n_j} \cdot \vec{r_j} \right]
	 d\tau_r \right|^2 \\
	 &= 1/(2\pi)^{3N} \left| \int_{\mathcal{B}^N} \Psi \exp \left[ -i \sum_{j = 1}^N \vec{k}_{n_j}
	 \cdot \vec{r_j} \right] d\tau_r \right|^2 d\tau_p
 \end{align*}
We define a new function,
\begin{equation}
	\Phi(p_{n_1}, p_{n_2}, \ldots, p_{n_N}) = \frac{1}{(2\pi)^{3N/2}} \int_{\mathcal{B}^N} \Psi \exp
	\left[ -i \sum_{j = 1}^N \vec{k}_{n_j} \cdot \vec{r_j} \right] d\tau_r
\end{equation}
Then,
\begin{equation}
	C_{n_1, n_2, \ldots, n_N} = (2\pi)^{3N/2} \frac{1}{V^N} \Phi(p_{n_1}, p_{n_2}, \ldots, p_{n_N})
	= \frac{1}{(2\pi)^{3N/2}} \Phi(p_{n_1}, p_{n_2}, \ldots, p_{n_N}) d\tau_p
\end{equation}
And,
\begin{equation}
	\Psi = \sum \Phi(p_{n_1}, p_{n_2}, \ldots, p_{n_N}) \exp \left[ i \sum_{j = 1}^N \frac{1}{\hbar}
	\vec{p}_{n_j} \cdot \vec{r_j} \right] d\tau_p
\end{equation}

Finally taking the limit as $x_0, y_0, z_0 \to -\infty$ and $L_x, L_y, L_z \to +\infty$ we get,
\begin{equation}
	\begin{aligned}
		\Psi = \frac{1}{(2\pi)^{3N/2}} \int_{\mathcal{R}^N} \Phi \exp \left[ i \sum_{j = 1}^N \frac{1}{\hbar} \vec{p}_j \cdot \vec{r_j} \right] d\tau_p \\
		\Phi = \frac{1}{(2\pi)^{3N/2}} \int_{\mathcal{R}^N} \Psi \exp \left[ -i \sum_{j = 1}^N \frac{1}{\hbar} \vec{p}_j \cdot \vec{r_j} \right] d\tau_r
	\end{aligned}
\end{equation}
With the probability density of momentum being $|\Phi|^2$.

\end{document}
