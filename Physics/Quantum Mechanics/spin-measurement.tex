\documentclass[a4paper, 12pt]{article}

\setlength{\parindent}{0pt}
\renewcommand{\indent}{\hspace{3ex}}

\usepackage[top=1in,bottom=1in,left=1.25in,right=1.25in]{geometry}
% \usepackage[document]{ragged2e}
\usepackage{fancyhdr}
\renewcommand{\headrulewidth}{0pt}
\renewcommand{\footrulewidth}{1.2pt}
\pagestyle{fancy}
\fancyhead{}
\fancyfoot{}
\fancyfoot[R]{page \thepage}

\renewcommand{\maketitle}{
	\begin{center}
		\Huge\bfseries{Spin Measurement in Any Direction (Spin 1/2)}
	\end{center}
	\bigskip
}

\usepackage{titlesec}
\titleformat{\section}
	{\normalfont\Large\bfseries}{\thesection}{1em}{}[{\titlerule}]
\titleformat{\subsection}
	{\normalfont\large\bfseries}{$\bullet$}{1ex}{}

\usepackage{hyperref}
\hypersetup{
  colorlinks   = true,
  urlcolor     = blue,
  linkcolor    = blue,
  citecolor    = red
}

\usepackage{common}
\usepackage{amsmath}
\usepackage{braket}
\usepackage{textcomp}

\newcommand{\Su}{\dfrac{\hbar}{2} \begin{pmatrix} \cos \gamma & \cos \alpha - i \cos \beta \\ \cos \alpha + i \cos \beta & - \cos \gamma \end{pmatrix}}

\begin{document}

\maketitle

\section{Introduction}

\indent The spin matrices for x, y, z axes are give by,
\begin{align}
	S_x = \frac{\hbar}{2} \begin{pmatrix}0 &1\\ 1 &0\end{pmatrix}
	&&S_y = \frac{\hbar}{2} \begin{pmatrix}0 &-i\\ i &0\end{pmatrix}
	&&S_z = \frac{\hbar}{2} \begin{pmatrix}1 &0\\ 0 &-1\end{pmatrix}
	\label{def: spin matrix}
\end{align}

\indent We want to measure spin along the direction $\vec u$. Where, $\vec u = \cos \alpha \, \hat i + \cos \beta \, \hat j + \cos \gamma \, \hat k$ is a unit vector described by it's directional cosines. Let us assume that the spin operator corresponding to this measurement is,
\begin{align*}
	S_u &= \cos \alpha \, S_x + \cos \beta \, S_y + \cos \gamma \, S_z \\
		&= \Su \labelthis{def: S_u}
\end{align*}


\section{Checking Consistency}

\indent For our assumption to be consistent $S_u$ needs to have two orthonormal eigenvectors with the eigenvalues $\pm \hbar/2$. If $\lambda$ is an eigenvalue of $S_u$ then,
\[ | S_u - \lambda \text{I} | = 0 \]
Putting the value of $S_u$ we get,
\begin{align*}
	\begin{vmatrix} \Su - \lambda \begin{pmatrix}1 &0\\ 0 &1\end{pmatrix} \end{vmatrix} &= 0 
\end{align*}
Let us define $\lambda' = 2/\hbar \, \lambda$. Then the above equation becomes,
\begin{align*}
	\begin{vmatrix} \cos \gamma - \lambda' & \cos \alpha - i \cos \beta \\ \cos \alpha + i \cos \beta & - \cos \gamma - \lambda'  \end{vmatrix} &= 0 \\
	\Rightarrow - \cos^2 \gamma + {\lambda'}^2 - \cos^2 \alpha - \cos^2 \beta &= 0
\end{align*}

Since $\vec u$ is an unit vector the above equation reduces to $-1 + {\lambda'}^2 = 0$. This means,
\begin{equation}
	\lambda = \pm \, \frac{\hbar}{2}
\end{equation}
Now $S_u$ is a linear sum of Hermitian Operators with real coefficients. Thus $S_u$ is also a hermitian operator. We can clearly see this from \eqref{def: S_u}. Meaning,
\begin{equation}
	{S_u}^\dagger = S_u
\end{equation}
We know, eigenvectors with distinct eigenvalues of a Hermitian operator are orthogonal to each other. Meaning $S_u$ has two orthonormal eigenstates with eigenvalues $\pm \hbar/2$. Meaning the definition of $S_u$ in \eqref{def: S_u} is consistent with the behavior of spin.


\section{Calculating Eigenvalues}
\indent Let us define the two eigenvectors of $S_u$ by $\ket{\uparrow}_u$ and $\ket{\downarrow}_u$ with eigenvalues $+ \, \hbar/2$ and $+ \, \hbar/2$ respectively. Then for $\ket{\uparrow}_u$,
\begin{align*}
	\Su \begin{pmatrix} a \\ b \end{pmatrix} &= \frac{\hbar}{2} \begin{pmatrix} a \\ b \end{pmatrix}
\end{align*}
This gives us the following equations,
\begin{align*}
	\cos \gamma \, a + (\cos \alpha - i \cos \beta) b = a\\
	(\cos \alpha + i \cos \beta) a - \cos \gamma \, b = b\\
\end{align*}
These two equations are the same equations. Solving them we will get,
\begin{equation}
	b = \frac{\cos \alpha + i \cos\beta}{1 + \cos \gamma} a
	\label{ex: b in terms of a}
\end{equation}
Since $\ket{\uparrow}_u$ has unit norm,
\begin{align*}
	\ket{\uparrow}_u^\dagger \ket{\uparrow}_u &= 1\\
	\Rightarrow \begin{pmatrix} a^* &b^* \end{pmatrix} \begin{pmatrix} a \\b \end{pmatrix} &= 1
\end{align*}
Putting the value of b,
\[ \left( 1 + \frac{\cos^2 \alpha + \cos^2 \beta}{(1 + \cos \gamma)^2} \right) |a|^2 = 1 \]
Thus,
\begin{align*}
	|a|^2 &= \frac{(1 + \cos \gamma)^2}{1 + \cos^2 \gamma + 2 \cos \gamma + \cos^2 \alpha + \cos^2 \beta}\\
		  &= \frac{(1 + \cos \gamma)^2}{1 + 1 + 2 \cos \gamma}\\
		  &= \frac{1 + \cos \gamma}{2}\\
		  &= \cos^2 (\gamma/2)
\end{align*}
We are going to pick the real value for $a$ as the phase does not matter in this case. Thus,
\begin{align*}
	\ket{\uparrow}_u &= \begin{pmatrix} \cos (\gamma/2) \\\\ \dfrac{(\cos\alpha + i\cos\beta)\cos (\gamma/2)}{1 + \cos \gamma} \end{pmatrix}\\
					 &= \begin{pmatrix} \cos (\gamma/2) \\\\ \dfrac{\cos \alpha + i \cos \beta}{2 \cos (\gamma/2)} \end{pmatrix} \labelthis{ex: up spin in u direction}
\end{align*}
For the down spin we will get $\cos \gamma - 1$ in the denominator in \eqref{ex: b in terms of a}. This will lead to $\sin \gamma/2$ in the expression of $\ket{\downarrow}_u$.
\begin{equation}
	\ket{\downarrow}_u = \begin{pmatrix} \sin (\gamma/2) \\\\ - \, \dfrac{\cos \alpha + i \cos \beta}{2 \sin (\gamma/2)} \end{pmatrix} \labelthis{ex: down spin in u direction}
\end{equation}


\section{Measurement}
\indent Now, let us assume that the particle was in up state, $\ket{\uparrow}_z$, in the z direction. We are going to measure the particles spin along a direction in the z-x plane. The direction makes angle $\theta$ with the z axis. Meaning $\gamma = \theta$, $\alpha = \pi/2 - \theta$ and $\beta = \pi / 2$. Putting these values in \eqref{ex: up spin in u direction} and \eqref{ex: down spin in u direction},
\begin{align*}
	\ket{\uparrow}_u = \begin{pmatrix} \cos (\theta/2) \\\\ \sin(\theta/2) \end{pmatrix} && \ket{\uparrow}_u = \begin{pmatrix} \sin (\theta/2) \\\\ - \cos(\theta/2) \end{pmatrix}
\end{align*}
Thus, the probability amplitude of $\ket{\uparrow}_u$ in $\ket{\uparrow}_z$ is,
\begin{equation*}
	\bra{\uparrow}_u \ket{\uparrow}_z = \begin{pmatrix} \cos (\theta/2) & \sin(\theta/2) \end{pmatrix} \begin{pmatrix} 1 \\ 0 \end{pmatrix} = \cos (\theta/2)
\end{equation*}
Similarly,
\begin{equation*}
	\bra{\downarrow}_u \ket{\uparrow}_z = \sin (\theta/2)
\end{equation*}
\indent Thus the probability of getting an up spin in $\vec u$ direction is $\cos^2(\theta/2)$ and for down spin it is $\sin^2 (\theta/2)$.

\section{Interpretation}
\indent We can interpret these results by saying that the angle in real becomes half that angle in spin space for spin 1/2 particles. For example the angle between angular momentum represented by up and down spin in real space is $180^\circ$. This may make it seem like up and down spin are just scalar multiple (-1) of each other. However in spin space they are orthogonal. This is because $180^\circ$ becomes $90^\circ$ degrees in spin space and dot product between vectors at $90^\circ$ is zero. Similarly we can use this interpretation to describe why up spin can be expressed as a linear sum of left and right spin. Because the angle in real space, $90^\circ$, becomes $45^\circ$ in spin space. Meaning up spin in z direction has equal parts of left and right spin (in x or y direction).

 
\end{document}
