\documentclass[a4paper, 11pt]{article}

\usepackage[a4paper, width=150mm, top=25mm, bottom=25mm, headheight=10mm]{geometry}
\usepackage{fancyhdr}
\renewcommand{\headrulewidth}{0pt}
\renewcommand{\footrulewidth}{1.2pt}
\pagestyle{fancy}
\fancyhead{}
\fancyhead[R]{\leftmark}
\fancyfoot{}
\fancyfoot[R]{page \thepage}
\fancypagestyle{sectionpagestyle}{
	\fancyhead{}
	\fancyfoot{}
	\fancyfoot[R]{page \thepage}
}

\usepackage{titlesec}
\titleformat{\section}
	{\normalfont\Large\bfseries}{\thesection}{1em}{}[{\titlerule}]

\usepackage{bookmark}
\usepackage{hyperref}
\hypersetup{
	colorlinks   = true,
	urlcolor     = black,
	linkcolor    = black,
	citecolor    = black,
	bookmarksnumbered,
	pdfencoding=auto,
	psdextra
}

\usepackage{amsmath}
\usepackage{braket}

\begin{document}

\begin{center}
	\Large\bfseries When is Energy Quantized
	\vspace{.15em}\hrule
\end{center}

In quantum mechanics energy of an eigenstate is determined by solving the time independent
Schrodinger's equation.
\begin{equation}
	H \Psi = E \Psi
\end{equation}
Solving this equation we will get different eigenfunctions for different values of the energy.
Normalization of the wavefunction results in energy having discrete values. If the eigenstates are
however not normalizable then we usually take the integral of the eigenstates and in this case the
energy becomes continuous. For example the bound states of hydrogen atom are normalizable and has
discrete energies but the states with non-negative energy are not normalizable and has continous
energy (for example in scattering theory). We will now prove this fact.
\medskip

Let $\Psi_0$ be the solution of Schrodinger's equation with energy $E_0$ for $q = q_0$. If the
energy is continous then we can express the energy as a function of some parameters $q = (q_1, q_2,
\ldots, q_n)$. Then,
\begin{equation}
	H \Psi(\vec{r}, q) = E(q) \Psi(\vec{r}, q) \label{eigenvalue eqn 2}
\end{equation}
We can expand $E$ and $\Psi$ around $E_0$ and $\Psi_0$ respectively as a taylor series expansion.
\begin{align}
	E(q) &= E_0 + \sum_i (\partial_i E)_{q_0} \, \Delta q_i + \sum_i \frac{1}{2} (\partial_i^2
	E)_{q_0} \, (\Delta q_i)^2 + \sum_{i \neq j} (\partial_i \partial_j E)_{q_0} \, \Delta q_i
	\Delta q_j + \ldots \label{taylor series E} \\
	\Psi( \vec{r}, q) &= \Psi_0 + \sum_i (\partial_i \Psi)_{q_0} \, \Delta q_i + \sum_i \frac{1}{2}
	(\partial_i^2 \Psi)_{q_0} \, (\Delta q_i)^2 + \sum_{i \neq j} (\partial_i \partial_j \Psi)_{q_0}
	\, \Delta q_i \Delta q_j + \ldots \label{taylor series Psi}
\end{align}
Where,
\begin{equation}
	\partial_i^m \partial_j^{n} \cdots \partial_k^{\, l} \, \mathcal{F} =
	\frac{\partial^m}{{\partial q_i}^m} \frac{\partial^n}{{\partial q_j}^n} \cdots
	\frac{\partial^l}{{\partial q_k}^l} \, \mathcal{F} = \mathcal{F}_{\, i j \ldots k}^{\, m n
	\ldots l}
\end{equation}
The hamiltonian is independent of $q$. Differentiating both sides of \eqref{eigenvalue eqn 2} we
get,
\begin{equation}
	H \Psi_{i j \ldots k}^{m n \ldots l} = E_{i j \ldots k}^{m n \ldots l} \Psi + E
	\Psi_{i j \ldots k}^{m n \ldots l}
\end{equation}
Evaluating this expression at $q = q_0$,
\begin{equation}
	H (\Psi_{i j \ldots k}^{m n \ldots l})_{q_0} = (E_{i j \ldots k}^{m n \ldots l})_{q_0} \Psi_0 +
	E_0 (\Psi_{i j \ldots k}^{m n \ldots l})_{q_0}
	\label{differentiating eigenvalue equation at q_0}
\end{equation}
If the wavefunction $\Psi_0$ is normalizable and it's derivatives go to zero fast enough we can
premultiply the equation above with ${\Psi_0}^*$ and take the integral over all space.
\begin{alignat*}{2}
	&\braket{ \Psi_0 | H (\Psi_{i j \ldots k}^{m n \ldots l})_{q_0} } &&= \braket{ \Psi_0 | (E_{i j
	\ldots k}^{m n \ldots l})_{q_0} \Psi_0 } + \braket{ \Psi_0 | E_0 (\Psi_{i j \ldots k}^{m n
	\ldots l})_{q_0} } \\
	\implies &\braket{ H \Psi_0 | (\Psi_{i j \ldots k}^{m n \ldots l})_{q_0} } &&= (E_{i j \ldots
	k}^{m n \ldots l})_{q_0} \braket{ \Psi_0 | \Psi_0 } + E_0 \braket{ \Psi_0 | (\Psi_{i j \ldots
	k}^{m n \ldots l})_{q_0} } \\
	\implies &\braket{ E_0 \Psi | (\Psi_{i j \ldots k}^{m n \ldots l})_{q_0} } &&= (E_{i j \ldots
	k}^{m n \ldots l})_{q_0} + E_0 \braket{ \Psi_0 | (\Psi_{i j \ldots k}^{m n \ldots l})_{q_0} } \\
	\implies &(E_{i j \ldots k}^{m n \ldots l})_{q_0} = 0
\end{alignat*}
We have now shown that all the derivatives of $E$ at $q_0$ is zero. Using this value in the taylor
series expansion in \eqref{taylor series E} we arrive at,
\begin{equation}
	E(q) = E_0
\end{equation}
Setting the value of the derivatives of $E$ to zero in \eqref{differentiating eigenvalue equation at
q_0},
\begin{equation}
	H (\Psi_{i j \ldots k}^{m n \ldots l})_{q_0} = E_0 (\Psi_{i j \ldots k}^{m n \ldots l})_{q_0}
\end{equation}
Which implies that $(\Psi_{i j \ldots k}^{m n \ldots l})_{q_0}$ belongs to the eigenspace for energy
eigenvalue $E_0$. Then according to \eqref{taylor series Psi} $\Psi(\vec{r}, q)$ does not leave the
eigenspace as we are simply doing a linear sum. Finally we have shown that energy cannot be
continous around normalizable wavefunctions.

\end{document}
